% PICO Lead: Alan Robinson
Bubble chambers present a scalable, background-discriminating technology for dark matter detection with the unique capability to operate with a broad variety of target materials, requiring only that the target be a fluid with a vapor pressure.  Multiple experimental efforts are using this flexibility to push dark matter sensitivity into parameter spaces inaccessible to other detection techniques.

The PICO Collaboration uses freon-filled bubble chambers for nuclear recoil detection in targets with high spin-dependent and low spin-independent cross-sections, allowing the exploration of orders-of-magnitude more dark matter parameter space before reaching the neutrino fog than can be achieved in Si, Ge, and noble-liquid targets.  There is strong physics motivation for freon bubble chambers out to kiloton-year exposures, exposures that are plausible with this technique thanks to its field-leading electron recoil rejection ($O$(1) in $10^{10}$ ER events misidentified as nuclear recoils~\cite{PICO:2019rsv}) and monolithic liquid target.  Past PICO experiments at SNOLAB have reached exposures of $\sim$3~ton-days~\cite{PICO:2019vsc}, including a zero-background (observed) ton-day exposure at 3.3-keV threshold~\cite{PICO:2017tgi}.  The PICO-500 experiment, funded by the Canada Foundation for Innovation, is projected, given conservative estimates for muon-induced neutron rejection, to reach an exposure of  ton-years on a C$_3$F$_8$ target by 2025, including 63 ton-days at 3.2-keV threshold, giving 3 expected solar neutrino CE$\nu$NS events, followed by 126 ton-days above the $^8$B CE$\nu$NS endpoint.  This results in a spin-dependent WIMP-proton sensitivity at the $10^{-42}$~cm$^2$ level, commensurate with the spin-dependent WIMP-neutron sensitivity expected from Generation-2 LXe-TPCs.  Unlike the Generation-2 LXe-TPCs, however, PICO-500's projection is still four-orders-of-magnitude \emph{above} the C$_3$F$_8$ neutrino fog.

A freon bubble chamber large enough to reach atmospheric neutrino sensitivity will require the development of a completely new detector ``inner vessel,'' i.e. the vessel containing the superheated liquid target.  Two factors limit the size of the synthetic silica glass jars used for PICO-500: no facilities exist to construct jars larger than those made for PICO-500, and if larger silica jars were to be constructued, the detector livetime would be limited by the alpha activity of synthetic silica.  The surface chemistry and smooth surface provided by silica glass will need to be replicated with a material capable of surface radioactivity of less than 5 nBq/cm$^2$ in order to construct a 50-ton detector sensitive to an atmospheric neutrino event within 5 years.  This level of radiopurity has been exceeded in the large surfaces of Kamland-ZEN \cite{Gando:2020ggh}.  Tests of similar materials are ongoing to determine their mechanical and surface suitability with published results expected by 2023.  Inner vessel replacement of an existing detector would then follow prior to proposing a kiloton-year scale experiment.
