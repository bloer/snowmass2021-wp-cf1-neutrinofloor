PICO Lead: Alan Robinson

The PICO Collaboration uses freon-filled bubble chambers for nuclear recoil detection in targets with high spin-dependent and low spin-independent cross-sections, allowing the exploration of orders-of-magnitude more dark matter parameter space before reaching the neutrino fog than can be achieved in Si, Ge, and noble-liquid targets.  There is strong physics motivation for freon bubble chambers out to kiloton-year exposures and beyond, exposures that are plausible with this technique thanks to its field-leading electron recoil rejection ($O$(1) in $10^{10}$ ER events misidentified as dark matter~\cite{PICO:2019rsv}) and monolithic liquid target.  Past PICO experiments at SNOLAB have reached exposures of $\sim$3~ton-days~\cite{PICO:2019vsc}, including a zero-background (observed) ton-day exposure at 3.3-keV threshold~\cite{PICO:2017tgi}.  The PICO-500 experiment, funded by the Canada Foundation for Innovation, will reach an exposure of XXX ton-years on a C$_3$F$_8$ target by 20YY, including XXX ton-years at ZZ-keV threshold, giving NN expected solar neutrino CE$\nu$NS events, followed by XXX ton-years above the $^8$B CE$\nu$NS endpoint.  This results in a spin-dependent WIMP-proton sensitivity commensurate with the spin-dependent WIMP-neutron sensitivity expected from Generation-2 LXe-TPCs.  Unlike the Generation-2 LXe-TPCs, however, PICO-500's projection is still four-orders-of-magnitude \emph{above} the C$_3$F$_8$ neutrino fog.

1-2 paragraphs (plus cross-reference to IF-08) discussion of instrumentation R\&D enabling the larger bubble chambers needed to explore this parameter space after PICO-500.

