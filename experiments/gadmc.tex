GADMC lead: Graham Giovanetti

The DarkSide-20k (DS-20k) experiment of the multinational Global Argon Dark Matter Collaboration (GADMC) will search for dark matter using a low-radioactivity, underground argon (UAr) target instrumented as a dual-phase time projection chamber (TPC)~\cite{Aalseth:2018gq}. The GADMC includes members from the ArDM, DarkSide-50, DEAP-3600, MiniCLEAN, and XENON collaborations, and DS-20k will utilize successful elements from these experiments. These include the use of UAr, which DarkSide-50 demonstrated to have an $^{39}$Ar concentration at least 1400 times smaller than atmospheric argon~\cite{Agnes:2015gu,Agnes:2016fz,Agnes:2018ep}, and a sealed, radiopure acrylic structure, a technology pioneered by the DEAP-3600 experiment~\cite{Nantais:2013jp,Amaudruz:2018gr}. The TPC is designed to take advantage of the favorable properties of liquid argon, including demonstrated electron recoil background discrimination power better than $10^8$~\cite{Adhikari:2021} and excellent chemical purity~\cite{Ajaj:2019jx,Ajaj:2019jk}, and operate with $<0.1$ background events within the 20.2~t fiducial volume over a ten year run, other than an expected ($3.2 \pm 0.6$) events from coherent neutrino scattering. The DS-20k experimental apparatus consists of three nested detectors installed within a membrane cryostat nearly identical to the two existing ProtoDUNE cryostats~\cite{Abi:2020gv,Abi:2020hj,Abi:2020je}: the inner dual- phase argon TPC, a neutron veto, and an outer muon veto. The apparatus will be located in Hall C of the Gran Sasso National Laboratory (LNGS). The DS-20k UAr target will be extracted by Urania~\cite{Aalseth:2018gq}, an argon extraction plant capable of extracting 330~kg/d of UAr, and purified with the Aria plant~\cite{Agnes:2021us}, a 350~m cryogenic distillation column designed to separate argon and other rare stable isotopes.

The ultimate objective of the GADMC is the construction of the Argo detector, which will have a 300~t fiducial mass and will push experimental sensitivity to the point at which the coherent scattering of atmospheric neutrinos becomes a limiting background. The excellent electron recoil (ER) rejection possible in argon will eliminate backgrounds from solar neutrinos, which will extend the sensitivity of Argo beyond that of technologies with more limited ER discrimination. Such a large detector would also have excellent sensitivity to a neutrino burst associated with a galactic supernova. If located at SNOLAB or at similar depth, Argo will also have the potential to observe CNO neutrinos for the first time and solve the Solar Metallicity Problem~\cite{Agnes:2020wd,Franco:2016ex}.

The further expansion of several technologies are critical for the success of Argo. The continued development and operation of Urania~\cite{Aalseth:2018gq} and Aria~\cite{Agnes:2021us} would enable 400~t of UAr to be extracted and purified over a period of about five years. Facilities for the long-term underground storage and assay of this argon are also needed. The Argo detector will be instrumented with more than 100~m$^2$ of photodetectors. This would be greatly simplified by the continued evolution of the large-area SiPM-based photosensors developed for DS-20k~\cite{DIncecco:2018hy,DIncecco:2018fx} into digital SiPMs, which would reduce the quantity of cables, significantly reduce noise, and decrease the background rate in the detector. Finally, all future detectors entering into the neutrino fog would benefit from improved atmospheric neutrino background modeling, which currently dominates the uncertainty on the experimental sensitivity.
