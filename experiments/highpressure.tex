Lead: Ben Monreal

When can a WIMP search use a gas target rather than a liquid?  Gas targets have some microphysical advantages, including: reduced ionization quenching for low-energy nuclear recoils; track-length-based recoil/beta discrimination; low-Fano-factor calorimetry that resolves x-ray backgrounds as lines. A major disadvantge of gas targets is their lack of self-shielding.  However, at the scales needed to reach the neutrino fog, we believe there a route to ultra-high-pressure gas TPCs.  If we can adopt petroleum-industry standard methods for storing large volumes of high pressure gas underground, we can build ultra-high-pressure WIMP targets (up to 500 bar)---big enough to begin self-shielding---with what we suspect to be low-cost instrumentation.  For one example of this new geometry, we describe a 10~m diameter, 80~m tall cylindrical target balloon, which we fill with 500~T of neon at 100~bar and operate as an inward-drift gas TPC; that configuration was optimized for neutrino physics but appears to offer powerful mid-range (2--20 GeV) dark matter sensitivity.  Using solution mining methods in a salt dome, a cavern big enough to host this (including 20~m of gas shielding) could be excavated for under US\$6M.  Many other options (different gases, pressures, sizes, geometries) appear worth exploring.  There are many physics, mechanical engineering, and cost uncertainties to this approach, but we argue that basic R\&D now will help us identify scalable, low-cost detector configurations using previously-impossible targets. 