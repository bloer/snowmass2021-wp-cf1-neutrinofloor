The Super Cryogenic Dark Matter Search (SuperCDMS) SNOLAB experiment is a 2$^{nd}$ generation dark matter experiment, which will commission its solid-state detectors 2 km underground at the SNOLAB facility in Sudbury, Canada.  The solid-state detectors are germanium or silicon crystals patterned with both Quasiparticle-assisted Electrothermal-feedback Transition edge sensors (QETs) as phonon sensors and electrodes used for charge readout and/or biasing the crystal.  The patterned QETs and electrodes are optimized to operate as either High Voltage (HV) or interleaved Z-dependent Ionization and Phonon (iZIP) detector.

In an HV detector, the electrodes are used to apply a relatively large bias voltage across the crystal.  The applied voltage gives rise to an \textbf{E}-field that causes electrons and holes to drift to opposing faces of the detector. As they move through the crystal, the charges scatter off the lattice, generating additional phonons via the Neganov-Trofimov-Luke (NTL) effect ~\cite{Neganov1985, Luke1988}.  The resulting total energy observed by the QETs on the detector faces is $E_{tot} = E_r + N_{eh}eV_{b}$, Where E$_r$ is the initial recoil energy, and N$_{eh}$ is the number of electron-hole pairs initially created. These devices have an improved ultra-high resolution and reach lower thresholds allowing them to probe lower DM masses.  The HV detectors are expected to explore WIMP DM masses down to ~0.3 GeV. 

In an iZIP detector, a low voltage bias is applied in order to minimize the NTL effect. The electrodes are also used as charge collectors.   iZIP detectors therefore measure both prompt phonons and ionization charge signals, which can be used to perform ER/NR discrimination based on the difference in ionization yield. Furthermore, optimization of the electrode layout allows identification of surface events via a charge signal collection asymmetry: bulk (nominally symmetric signals) and surface (highly asymmetric signals) events.  This allows rejection of beta particles and further reduces the background in the operation of these devices. The advanced rejection capabilities of these devices project sensitivities in a ``background-free'' mode to WIMPs with masses $>$5 GeV and in a “limited-discrimination” mode to WIMPS $>$1 GeV~\cite{SCDMS2017}.

The SuperCDMS SNOLAB experiment anticipates observing approximately 50 events associated with neutrino interactions, but will not reach the neutrino fog.  The background and detector improvements needed to reach the fog have been identified as part of a near-term SuperCDMS upgrade plan.  Backgrounds improvements include sourcing new material, replacing components with lower background alternatives, and improving detector fabrication/tower assembly to reduce the $^{210}$Pb plated onto the surface from radon, all within a reasonable cost for implementation.  Proposed detector upgrades common to both HV and iZIP style detectors include: 1) smaller detector sizes; 2) lowering the TES critical temperature (T$_C$); and 3) improving phonon transmission across interfaces.  Scaling to smaller detectors will improve the phonon/ionization resolution of the SuperCDMS detectors, which in turn will improve rejection of bulk ER backgrounds. This development is reasonably mature with prototype Si HVeV detectors (an HV style device) already deployed at test facilities ~\cite{SCDMS2019,SCDMS2020}.  R\&D efforts are set to shift toward the development of a Ge HVeV detector and optimization of these devices.  Lowering of the TES T$_C$ is less mature, but a viable path forward is known. The thin tungsten film that is at the heart of the QET can grow in two different phases, $\alpha$-W with a T$_C$ of 15 mK or $\beta$-W with a T$_C$ over 2 K, during deposition. By mixing these two phases in different ratios the T$_C$ can be tuned between the two extremes.  The challenge is in identifying the proper deposition parameters under which the W film will grow at the proper ratio in a reproducible and controllable manner.  Improvements in phonon transmission from the crystal all the way to the W TES is the least mature advancement being considered, with no immediate avenues forward identified.