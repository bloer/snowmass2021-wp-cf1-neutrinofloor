% SBC Lead: Eric Dahl

The Scintillating Bubble Chamber (SBC) Collaboration uses liquid-noble bubble chambers to extend the nuclear/electron recoil discrimination capability of freon-filled chambers to the sub-keV thresholds needed to reach the neutrino fog at 1~GeV.  Liquid-noble bubble chambers differ from their freon-filled cousins in two (related) ways. First, the scintillation coincident with bubble nucleation in the target allows event-by-event energy reconstruction~\cite{Baxter:2017ozv}, which can be used to reject of backgrounds above the few-keV scintillation detection threshold.  Second, noble liquids can be superheated to a far greater degree than molecular fluids, showing sensitivity to sub-keV nuclear recoils while remaining completely insensitive to electron-recoil backgrounds~\cite{Durnford:2021cvb}.  The combination of scalability, low threshold, and background discrimination \emph{at low threshold} gives the noble-liquid bubble chamber unique capability to explore the neutrino fog in the 1--10~GeV WIMP mass range.

The ultimate threshold reach of the liquid-noble bubble chamber is not yet known.  A 10-kg LAr bubble chamber (also capable of operation with LXe) has been built at Fermilab to resolve this question, designed to probe thermodynamic thresholds as low as 40~eV and accomplish nuclear recoil sensitivity calibrations with $O$(10)-eV resolution.  This device is now being commissioned, and low-threshold calibrations will begin in 2023.  Meanwhile, the Canada Foundation for Innovation has funded the construction of a twin 10-kg device for SBC's first dark matter search \cite{Giampa:2021wte}, which has received GW1 approval at SNOLAB.  At a 100-eV nuclear recoil detection threshold (SBC's benchmark until the calibration campaign is complete), this 10-kg LAr chamber will observe 2.5 solar neutrino CE$\nu$NS events per live year.  A follow-up 100-eV threshold LAr experiment at the scale of PICO-500 (1-ton-year exposure) will be sufficient to explore the neutrino fog to $n=2$ at 1-GeV WIMP mass. 
