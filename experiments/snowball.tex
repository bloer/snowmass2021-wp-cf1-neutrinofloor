The snowball chamber is a nascent technology which is analogous in operational principles to superheating in bubble chambers and supersaturation in cloud chambers, except that it relies on supercooling~\cite{szydagis2021}.  The first prototype was constructed by Profs. Levy and Szydagis with students at UAlbany SUNY and shown to likely be sensitive to nuclear recoils from neutrons as the radioactive calibration source, with low sensitivity to electron recoils, as in dark matter bubble chambers~\cite{PICO:2019vsc}. The detector relies on lowering the temperature of liquid water below its freezing point in a sufficiently clean and smooth container so that it becomes metastable instead of immediately solidifying.  An incoming particle such as a dark matter WIMP should be able to trigger the phase transition, and potentially encode directionality as well via the intense hydrogen bonding of water.  Advantages of this technique would be the potential for sub-keV energy threshold and the use of water (for scalability, ease of purification, background neutron moderation, and excellent spin-dependent-proton sensitivity).  If the threshold is indeed as low as claimed for decades for supercooled water in atmospheric sciences, then even only a few kg deployed underground for only a few years could lead to world-leading sub-GeV limits for both the standard SI and SD-proton operators.

In order for this detector technology to become viable, numerous challenges will need to be overcome.  The water volume will need to be sufficiently purified in order to achieve a low energy threshold by lowering the temperature sufficiently without nucleation sites present; reduction in background nucleation through these mitigations, combined with faster reheating methods post-event, should lead to the required livetime of $>$50\%; calibrations of backgrounds from all sources will need to be performed, using betas and gamma-rays to fully characterize the electronic recoil discrimination power, as well as alphas to determine how much Radon contamination would be an issue, and these calibrations would need to be performed as a function of temperature (and pressure) with the goal of finding a “sweet spot” temperature.

With most of the materials, equipment, and supplies exist already through earlier seed-funding initiatives, even for building a larger-scale (only grams tested thus far) viable dark matter experiment that is ready for underground deployment, the required resources for construction would be limited, and the funding agencies will need to take a risk on a new idea that is not only an extrapolation from existing concepts: the cost is low so the risk is low, but the return high, based on the advantages discussed earlier.