Work is underway to explore the feasibility of a low background kTon-scale liquid argon time projection chamber in the context of the Deep Underground Neutrino Experiment (DUNE).  The DUNE program consists of modules, and though the designs for the first two modules have been selected, the third and fourth (the `modules of opportunity’) remain to be determined.  A recent community effort~\cite{SnowmassLowBgDUNElikeWP} explored the option of making one of these a dedicated low background module, with possible sensitivity to high mass WIMPs~\cite{Church_2020}.  This detector could also confirm a galactic WIMP signal discovered in the generation two detectors using an annual modulation.  

This DUNE-like detector would adapt the standard vertical drift design, with the addition of an optically isolated inner volume where increased light detection allows improved energy resolution at low energies and pulse shape discrimination for background reduction.  The detector would take advantage of the significant self-shielding of the liquid argon to reduce the backgrounds in a 3 kton fiducial volume (compared to 10 ktons in the full module).  Use of low radioactivity underground argon reduces the argon-39 and argon-42 internal backgrounds.

The primary research and development challenges for this detector design are associated with the large scale; in general, the radioactive background requirements are less strict than dedicated dark matter experiments.  Quality control of an assay program will need to be strict to ensure the large amount of material meets requirements.  Radon purification and emanation control in large amounts of liquid argon will need to be demonstrated.  Cleanliness controls for a large detector assembled underground will need to be developed and demonstrated.  Current known underground argon sources are not large enough to supply a detector of this size, and the collaboration is in discussion with commercial gas producers to determine whether a cost-effective supply can be achieved.  The main engineering challenges of this detector are associated with the production and installation of a large amount of SiPMs required for instrumenting the detector to reach the required energy threshold.