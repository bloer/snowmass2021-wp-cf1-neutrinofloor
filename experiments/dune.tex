DUNE-like Lead: Chris Jackson

Work is underway to explore the feasibility of a low background kTon-scale liquid argon time projection chamber. The Deep Underground Neutrino Experiment (DUNE) is a detector currently under construction primarily focused on precision neutrino oscillation measurements, searches for physics beyond the standard model and neutrinos from supernova sources. This detector will consist of four modules, and though the designs for the first two modules have been selected, the third and fourth (the `modules of opportunity’) remain to be determined. A recent community effort [ref white paper] has begun to explore the option of making one of these a dedicated low background module, with increased physics sensitivity to low energy neutrino supernova and solar sources. A study has shown that this detector could have sensitivity to high mass WIMPs [ref dark matter paper], comparable to the coming generation of detectors but on a faster timescale. This detector could also confirm a signal discovered in the generation two detectors using an annual modulation.  

This DUNE-like detector is planned to adapt the standard vertical drift design, with the addition of an optically isolated inner volume where increased light detection allows improved energy resolution at low energies and pulse shape discrimination for background reduction. The detector takes advantage of the significant self-shielding of the liquid argon to reduce the backgrounds in a 3 kton fiducial volume (compared to 10 ktons in the full module). The addition of a thin water shield reduces the external neutron backgrounds by 3 orders of magnitude. A materials and assay program reduces the internal backgrounds by a similar amount, though self-shielding means the requirements are not as strict as those expected at generation 2 experiments. Active purification reduces radon within the detector volume. Use of low radioactivity underground argon reduces the argon-39 and argon-42 internal backgrounds.

The primary research and development challenges for this detector design are associated with the large scale; in general, the radioactive background requirements are less strict than dedicated dark matter experiments. Quality control of an assay program will need to be strict to ensure the large amount of material (for example, 1 kton of stainless steel in the cryostat) meets requirements and systems will need to be designed to ensure this is achieved through a large, distributed assay program. Radon purification and emanation control in large amounts of liquid argon will need to be demonstrated. Cleanliness controls for a large detector assembled underground will need to be developed and demonstrated. Current known underground argon sources are not large enough to supply a detector of this size. PNNL is in discussion with commercial gas producers to determine whether a cost-effective supply can be achieved. The main engineering challenges of this detector are associated with the production and installation of a large amount of SiPMs required to increase the light collection efficiency.

