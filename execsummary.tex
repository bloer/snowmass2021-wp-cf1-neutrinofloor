\newpage
\section*{Executive Summary}
\label{sec:execsummary}
\begin{itemize}
    \item There are multiple well-motivated dark matter candidates remaining in the ``traditional'' $\sim$GeV-scale mass range. The currently-funded suite of searches in this mass range (``Generation 2'' experiments) will not have sensitivity to fully test the majority of these candidates. 
    \item A ``Generation 3'' suite of experiments with an order of magnitude larger exposure would be able to fully test some candidates. Such a suite should include searches for spin-dependent interactions, which can uniquely test some models not probed by spin-0 targets. 
    \item Despite the maturity of the field, novel technologies should not be neglected. Current R\&D on new techniques will improve established  detector performance and provide new methods to mitigate backgrounds and probe complementary parameter space near the neutrino fog. 
    \item Substantial well-motivated parameter space will yet remain if dark matter signals are not observed by the G3 experiments. Irreducible neutrino backgrounds will cause substantially diminished returns on further increases in exposure. However, if the uncertainties in the neutrino fluxes are reduced, further increases may become feasible. Moreover, light, spin-dependent targets such as fluorine have substantially lower neutrino backgrounds and can therefore scale to larger masses even with current neutrino flux uncertainties. 
    \item Directional detectors are one possible way to reject neutrino backgrounds and thereby reach beyond the neutrino-limited point of current technology. Such detectors will require substantial R\&D investment to reach the size and level of background control required to explore this parameter space. Gas TPCs with micropattern gaseous detector (MPGD) readout should be advanced to the 10 m$^3$ scale. 
\end{itemize}