\section{Introduction}
\label{sec:intro}
In this paper, we consider the current and future state of direct experimental searches for interactions of nuclei with dark matter masses approximately in the GeV to TeV range. This mass range, roughly corresponding to what has traditionally been labeled Weakly Interacting Massive Particle (WIMP) dark matter,  has historically received the most focus. Consequently, there is a broad competitive landscape of different targets and mature technologies. 
Complementary papers cover lighter~\cite{SnowmassCF1WP2} and heavier~\cite{SnowmassCF1WP8} direct detection searches as well as indirect searches via astronomical observation~\cite{SnowmassCF1WP5}. 

Sensitivity in this mass range does not depend strongly on experiment threshold. Rather, improvements in sensitivity are driven almost entirely by increasing exposure (i.e., target mass). Even for mature technologies that have already fielded multiple generations of detectors, scaling up the detector size presents numerous technical challenges. The most significant shared challenge is that backgrounds passing all fiducial and analysis cuts, both from radioactivity and instrument noise, must decrease proportionally with detector mass. Similarly calibration of detector response functions must improve with every successive generation. More details of these common concerns are presented in \cite{SnowmassCF1WP3}. Other challenges are unique to a given detector technology, such as affordably maintaining high signal collection efficiency while covering a larger area, further from the fiducial volume. To ensure success in further searches, R\&D investments both addressing how to scale up existing technologies as well as examining totally new technologies are warranted. 

If not limited by other factors, dark matter detectors will be limited by irreducible backgrounds from neutrinos~\cite{monroe2007, strigari2009, billard2014b}. Originally dubbed the ``neutrino floor,''  the community now promotes the term ``neutrino fog,'' to better indicate that, rather than a hard limit on direct detection sensitivity, the neutrino background imposes a gradual penalty that can be overcome, at least to some extent. In Section~\ref{sec:neutrinofloor} we quantify the neutrino background's effect on detector sensitivity versus exposure. 

In Section~\ref{sec:theory} we provide a survey of theoretically-motivated dark matter candidates in this mass range consistent with experimental evidence from dark matter searches, collider experiments, and astronomical observatories. As we shall show, this mass range is well-covered by predictive theories, down to and well into the neutrino fog. Very few of these theories will be fully tested by the upcoming generation of funded experiments. As described in Section~\ref{sec:currentexperiments}, the \textit{subsequent} generation of current experiments (``generation 3'') is expected to reach deep enough into the neutrino fog to incur significant diminishing returns on further growth.  Further exploration of this theoretically-motivated parameter space will require either extreme scaling of detector size or developing new technology less sensitive to neutrino backgrounds, as discussed in Section~\ref{sec:beyondnufog}.


